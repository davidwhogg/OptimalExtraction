\documentclass[12pt, letterpaper]{article}

\addtolength{\topmargin}{-0.75in}
\addtolength{\textheight}{1.50in}
\setlength{\parindent}{1.1\baselineskip}

\begin{document}\sloppy\sloppypar\raggedbottom\frenchspacing

\section*{A think-o in spectroscopic optimal extraction}

\noindent
\textbf{David W. Hogg}\\
{\footnotesize
\textsl{Center for Cosmology and Particle Physics, New York University}\\
\textsl{Max-Planck-Institut f\"ur Astronomie}\\
\textsl{Flatiron Institute, a division of the Simons Foundation}
}

\paragraph{Abstract:}
The classic method of \emph{optimal extraction} for obtaining
one-dimensional spectra from two-dimensional spectrograph images
operates by constructing an ``optimal'' (in a certain sense) weighted
sum of sky-subtracted pixel values.
This optimal sum looks like a regression or a matched filter;
that's a good outcome, and correct.
The original paper, however, relies on two related, unnecessary, and
at least slightly incorrect, assumptions.
The first is that the data minus the sky divided by the point-spread
function will deliver an unbiased estimate of the true spectral
intensity.
The second is that the instrument PSF has a certain normalization
property, which requires that the data have been perfectly
flat-fielded, and that the associated noise estimates were adjusted
correctly in response.
Oddly, neither of these assumptions is required; both can be dropped.
The new expression for optimal extraction is very similar to the
original formula.
However it is simpler, less sensitive to uncertainties in the
flat-field, and correct under weaker assumptions.
If spectroscopy projects adopt the new expression presented here, it
will improve and simplify their data-analysis pipelines.
You're welcome.

\section{Optimal extraction: What's wrong with it?}

In the original paper (Horne, 1986), but modifying some (slightly hard
to understand) indexing, the optimal-extraction algorithm is given as
\begin{equation}
f_\lambda \leftarrow \frac{\sum_j W_{j\lambda}\,(D_j - S_j) / P_{j\lambda}}%
                          {\sum_j W_{j\lambda}}
\quad ,
\end{equation}
where $f_\lambda$ is an estimate of the flux of the source at
wavelength $\lambda$, the $W_{j\lambda}$ are ``optimal'' weights we
will give in a moment, the $D_j$ are the pixel intensities in pixels
$j$, the $S_j$ are sky intensities in pixels $j$ (presumed precisely
and accurately known, hahaha), and the $P_{j\lambda}$ represents the
point-spread function.
The weights $W_{j\lambda}$ that minimize variance are given by
\begin{equation}
W_{j\lambda} = \frac{P_{j\lambda}^2}{V_j}
\quad ,
\end{equation}
where $V_j$ is the variance of the noise in pixel $j$ (which,
presumably, is supposed to be the sum of the noise in the data and the
sky estimate.
And, by assumption, the point-spread function obeys a normalization
condition
\begin{equation}
1 = \sum P_{j\lambda}
\quad ,
\end{equation}
which makes the definition of $P_{j\lambda}$ the probability of a
\emph{detected} photon of wavelength $\lambda$ being detected in pixel
$j$.
Note that this is to be distinguished from the probability of a photon
of wavelength $\lambda$ at the top of the atmosphere being detected in
pixel $j$.
That is, $P_{j\lambda}$ is effectively the PSF in perfectly
flat-fielded data.

I would change the notation to the following ...

In my notation, the Horne optimal extraction algorithm becomes ...

The algorithm is justified by saying that the XXX are unbiased
estimates of YYY, and the weights deliver the minimum-variance
weighted mean of those unbiased estimates.

Unfortunately, this is not true for several reasons, all of which
relate to the fact that you can't know your point-spread function or
your flat-field accurately in locations in your spectrograph that are
not well illuminated by your object or your calibration lamps.

When you don't know this precisely, either the flat-fielding goes
wrong, or else the division by the point-spread function delivers
biased estimates of the intensity, or the correction of the data
uncertainties spoils the variance minimization, or all of the above.

What I will show (below) is that if you do everything consistently
between the flat-fielding and the data-uncertainty estimation, the
Horne expressions are not incorrect, they are just numerically
unstable.

But there is no need to introduce so opportunities for inconsistencies
or numerical instability when they aren't needed.

A more general statement is that you don't want to be correcting your
data for the flat-field; you want to be correcting your model for the
flat-field.
And in particular, you don't want to be dividing your data by tiny
numbers!

\section{The new optimal extraction}

There are proofs in statistics that the minimum-variance unbiased
estimator of a quantity is also the maximum-likelihood value for that
quantity, given the data.

...

\section{Extensions and discussion}

The point about covariant noise.

The point that they sky and the source should be modeled
simultaneously, not subtracted.

The point that the sky, the source, and the data used to determine the
PSF should all be modeled simultaneously!

And the flat, dark, and so on. Why not? If precision requirements are
high!

\section*{References}
\begin{list}{}{%
\rightmargin=0in
\leftmargin=\parindent
\itemindent=-1.0\leftmargin
\listparindent=0.0\leftmargin
}
\item Horne, K., 1986,
An optimal extraction algorithm for CCD spectroscopy,
PASP 98 609.
\end{list}

\end{document}
