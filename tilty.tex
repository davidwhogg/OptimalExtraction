\documentclass[12pt, letterpaper]{article}

% typesetting
\newcommand{\foreign}[1]{\textsl{#1}}
\newcommand{\etal}{\foreign{et al.}}

% page formatting
\addtolength{\topmargin}{-0.75in}
\addtolength{\textheight}{1.50in}
\setlength{\parindent}{1.1\baselineskip}
\sloppy\sloppypar\raggedbottom\frenchspacing

\begin{document}

\section*{\raggedright%
Spectroscopic optimal extraction when the line-spread function is slightly tilted}

\noindent
\textbf{David W. Hogg}\\
{\footnotesize
\textsl{Center for Cosmology and Particle Physics, New York University}\\
\textsl{Max-Planck-Institut f\"ur Astronomie}\\
\textsl{Flatiron Institute, a division of the Simons Foundation}
}

\paragraph{Abstract:}
The classic method of \emph{optimal extraction} for obtaining
one-dimensional spectra from two-dimensional spectrograph images
operates by constructing an ``optimal'' (in a certain sense) weighted
sum of sky-subtracted pixel values.
This optimal sum looks like a regression or a matched filter;
that's a good outcome, and correct.
HOGG...

\section{Why optimal extraction is a good idea.}

In the original paper (Horne, 1986)...

It can be re-cast as maximum-likelihood in a forward model...

In what follows, we are going to assume you are sensible and are using
the flat-relative form of optimal extraction (Zechmeister \etal,
2014), which is duh...

It gets around the deconvolve / reconvolve problems of other
forward-modeling methods...

But it is flummoxed if the LSF isn't axis-aligned with the
CCD pixel grid...

\section{A slightly tilty spectrograph}

We are going to re-write FROE (Zechmeister \etal, 2014) as
\begin{equation}
I(x,y) = r(x)\,f(x, y) + \mbox{noise}
\end{equation}
where
$I(x,y)$ is the data (photon counts in the sky-subtracted image maybe)
at the pixel located at 2-d detector position $(x,y)$,
$r(x)$ is the ratio of the object flux density to the flat flux density
at the wavelength corresponding to detector $x$-position $x$, and
$f(x,y)$ is the expected photon counts from the flat in the pixel
located at $(x,y)$.
The values of $f(x,y)$ are presumed to be known both accurately and
precisely from copious amounts of flat-field data.
The ratios $r(x)$ are learned by least-squares fitting to the data
$I(x,y)$.

We are going to modify this by permitting the wavelength direction $x$
to be distorted as a function of position in the detector.
$\Delta x$ is a polynomial in x, linear in y, or really y relative to
the trace ``center''.
\begin{equation}
I(x,y) = r(x + \Delta x)\,f(x, y)
\end{equation}
\begin{equation}
\Delta x = g(x)\,[y - y_c(x)]
\end{equation}
where, ...
Note that $g(x)$ can change sign and so on...

\section*{References}
\begin{list}{}{%
\rightmargin=0in
\leftmargin=\parindent
\itemindent=-1.0\leftmargin
\listparindent=0.0\leftmargin
\itemsep=0ex
\parsep=0ex
}
\item Horne, K., 1986,
An optimal extraction algorithm for CCD spectroscopy,
PASP 98 609.
\item Zechmeister, M., Anglada-Escud\'e, G., Reiners, A., 2014,
Flat-relative optimal extraction. A quick and efficient algorithm for stabilised spectrographs,
A\&A 561 59.
\end{list}

\end{document}
