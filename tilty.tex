\documentclass[12pt, letterpaper]{article}

\addtolength{\topmargin}{-0.75in}
\addtolength{\textheight}{1.50in}
\setlength{\parindent}{1.1\baselineskip}

\begin{document}\sloppy\sloppypar\raggedbottom\frenchspacing

\section*{Optimal extraction when the LSF is slightly tilted}

\noindent
\textbf{David W. Hogg}\\
{\footnotesize
\textsl{Center for Cosmology and Particle Physics, New York University}\\
\textsl{Max-Planck-Institut f\"ur Astronomie}\\
\textsl{Flatiron Institute, a division of the Simons Foundation}
}

\paragraph{Abstract:}
The classic method of \emph{optimal extraction} for obtaining
one-dimensional spectra from two-dimensional spectrograph images
operates by constructing an ``optimal'' (in a certain sense) weighted
sum of sky-subtracted pixel values.
This optimal sum looks like a regression or a matched filter;
that's a good outcome, and correct.
HOGG...

\section{Why optimal extraction is a good idea.}

In the original paper (Horne, 1986)...

\section*{References}
\begin{list}{}{%
\rightmargin=0in
\leftmargin=\parindent
\itemindent=-1.0\leftmargin
\listparindent=0.0\leftmargin
}
\item Horne, K., 1986,
An optimal extraction algorithm for CCD spectroscopy,
PASP 98 609.
\end{list}

\end{document}
